% !TEX root = ../thesis.tex

\chapter{Literature Review and Theoretical Foundations} \label{chp:literature}

% Chapter 2 Outline:
% 2.1 Large Language Models as Decision-Making Agents
% 2.2 Digital Twins: From Engineering Monitoring to Cognitive Media
% 2.3 Cognitive Architectures and Embodied Intelligence
% 2.4 Integration Challenges and Current Approaches
% 2.5 Research Gap Analysis and CORTEX Positioning

\section{Large Language Models as Decision-Making Agents}

\subsection{Evolution from Language Models to Autonomous Agents}
% - Historical development from statistical to neural language models
% - Transformer architecture and attention mechanisms
% - Emergence of few-shot learning and in-context adaptation
% - Scaling laws and emergent capabilities

\subsection{Tool-Augmented LLMs and Multi-Step Reasoning}
% - Chain-of-Thought and related prompting techniques
% - Tool learning and API integration (Toolformer, WebGPT)
% - Multi-agent frameworks and collaborative reasoning
% - Planning and execution frameworks (ReAct, AutoGPT)

\subsection{Current Limitations in Physical World Interaction}
% - Symbol grounding challenges in embodied scenarios
% - Temporal reasoning and state tracking limitations
% - Safety and reliability concerns in physical systems
% - Evaluation challenges for real-world performance

\section{Digital Twins: From Engineering Monitoring to Cognitive Media}

\subsection{Traditional Digital Twin Applications}
% - Manufacturing and industrial automation
% - Smart city and infrastructure management
% - Healthcare and biomedical applications
% - Aerospace and automotive industries

\subsection{Computational Architectures and Implementation}
% - Real-time data integration and processing
% - Multi-fidelity modeling approaches
% - Simulation and prediction capabilities
% - Update mechanisms and model calibration

\subsection{Toward Cognitive Digital Twins}
% - AI-enhanced digital twins
% - Natural language interfaces for digital twins
% - Reasoning and decision-making capabilities
% - Human-twin interaction paradigms

\section{Cognitive Architectures and Embodied Intelligence}

\subsection{Classical Cognitive Architectures}
% - SOAR, ACT-R, and symbolic approaches
% - Hybrid architectures combining symbolic and connectionist elements
% - Cognitive cycles and perception-action loops
% - Learning and adaptation mechanisms

\subsection{Modern Approaches to Embodied AI}
% - Sensorimotor integration and grounding
% - Reactive vs. deliberative control
% - World models and predictive processing
% - Social and collaborative embodied agents

\subsection{LLM-Based Cognitive Systems}
% - Language as a medium for reasoning and planning
% - Multimodal integration in LLM-based systems
% - Memory and context management in conversational agents
% - Alignment and safety in autonomous cognitive systems

\section{Integration Challenges and Current Approaches}

\subsection{Bridging Symbolic and Subsymbolic Processing}
% - Neuro-symbolic integration approaches
% - Knowledge representation and reasoning
% - Learning symbolic knowledge from data
% - Compositional generalization challenges

\subsection{Real-Time Requirements and Computational Constraints}
% - Latency considerations in interactive systems
% - Resource allocation and optimization
% - Edge computing and distributed processing
% - Trade-offs between accuracy and efficiency

\subsection{Evaluation and Benchmarking}
% - Metrics for physical world interaction
% - Simulation vs. real-world evaluation
% - Safety and robustness assessment
% - Generalization and transfer capabilities

\section{Research Gap Analysis and CORTEX Positioning}

\subsection{Identified Gaps in Current Literature}
% - Lack of systematic approaches to LLM-physical world integration
% - Limited exploration of Digital Twins as cognitive media
% - Insufficient attention to multi-domain generalization
% - Absence of comprehensive evaluation frameworks

\subsection{CORTEX's Unique Contributions}
% - Novel cognitive architecture design
% - Systematic Digital Twin integration approach
% - Cross-domain validation methodology
% - Comprehensive evaluation framework

\subsection{Chapter Summary}
% - Synthesis of reviewed literature
% - Positioning of CORTEX within the research landscape
% - Foundation for the proposed approach

% TODO: Add detailed content for each section
% Current status: Outline completed, detailed content to be developed
% Target completion: Before Candidacy Examination (July 2025)