% !TEX root = ../thesis.tex

\chapter{Literature Review} \label{chp:literature}

\section{Digital Twins: Maturity Models and Functional Gaps}

The concept of Digital Twins has evolved from early computer-aided design (CAD) and simulation tools into sophisticated real-time representations of physical systems. This evolution can be traced through several key developmental phases, each characterized by increasing levels of sophistication and integration.

Early Digital Twin concepts emerged from the aerospace and automotive industries, where complex systems required detailed virtual representations for design, testing, and maintenance purposes \cite{grieves2014digital}. These initial implementations focused primarily on geometric modeling and basic simulation capabilities, providing static representations that could be updated manually based on physical system changes.

The integration of Internet of Things (IoT) technologies marked a significant advancement in Digital Twin capabilities, enabling real-time data connectivity between physical and virtual representations \cite{qi2021digital}. This development transformed Digital Twins from static models into dynamic systems capable of continuous updates and real-time monitoring.

Current Digital Twin maturity models typically focus on engineering implementation details such as data connectivity, model fidelity, and update frequency \cite{jones2020characterising, lu2020digital}. While these frameworks provide valuable guidance for technical implementation, they often overlook the cognitive and decision-making aspects that are crucial for intelligent system operation.

The ISO 23247 standard provides a comprehensive framework for Digital Twin implementation, defining four key components: observable manufacturing element, Digital Twin, user, and digital twin network \cite{ISO23247}. However, this standard primarily addresses data integration and interoperability challenges rather than intelligent reasoning and autonomous decision-making capabilities.

Engineering maturity models have been developed to assess Digital Twin implementation quality and capabilities. The five-level maturity model proposed by various researchers categorizes implementations from basic descriptive models to fully autonomous predictive systems \cite{rasheed2020digital, tao2019digital}. While these models provide useful benchmarks for technical development, they do not adequately address the cognitive reasoning capabilities needed for intelligent system operation.

Functional classification systems attempt to categorize Digital Twins based on their intended purposes and capabilities. Common categories include descriptive twins (monitoring and visualization), diagnostic twins (fault detection and analysis), predictive twins (forecasting and planning), and prescriptive twins (optimization and control) \cite{kritzinger2018digital, negri2017review}. However, these classifications often focus on technical functionality rather than the complexity of decision-making tasks and reasoning requirements.

Current functional classification approaches have several limitations when applied to LLM-enhanced systems. They typically assume human interpretation of results, do not account for autonomous reasoning requirements, and lack systematic evaluation frameworks for cognitive capabilities. These gaps highlight the need for new classification approaches that consider the decision-making complexity and cognitive requirements of intelligent CPS.

\section{Large Language Models and the Grounding Problem}

The emergence of large language models (LLMs) has revolutionized natural language processing and demonstrated remarkable capabilities in reasoning, problem-solving, and knowledge synthesis \cite{brown2020language, chowdhery2022palm, openai2023gpt4}. These models show particular strength in tasks requiring common sense reasoning, complex language understanding, and multi-step problem solving.

Retrieval-Augmented Generation (RAG) architectures represent a significant advancement in addressing the knowledge limitations of LLMs \cite{lewis2020retrieval, karpukhin2020dense}. These approaches combine the reasoning capabilities of LLMs with external knowledge sources, enabling more accurate and up-to-date responses while reducing hallucination problems.

Recent developments in RAG implementations have explored various approaches to knowledge integration, including dense passage retrieval, sparse retrieval methods, and hybrid approaches \cite{khattab2020colbert, karpukhin2020dense, xiong2020approximate}. These methods demonstrate effectiveness in information retrieval tasks but face challenges when applied to structured data and real-time sensor information typical in CPS applications.

Advanced RAG architectures have been developed to handle more complex reasoning tasks, including multi-hop reasoning, fact verification, and knowledge graph integration \cite{petroni2021kilt, yasunaga2021qa}. However, these approaches typically focus on text-based knowledge and may not directly transfer to the heterogeneous data environments typical of physical systems.

The grounding problem in physical environments presents unique challenges that differ significantly from text-based applications \cite{harnad1990symbol, barsalou2008grounded}. Physical systems involve continuous processes, real-time constraints, and multi-modal data that require specialized approaches for effective LLM integration.

Paradigm mismatch occurs when LLM reasoning patterns, optimized for text-based tasks, are applied to physical world problems that require different types of reasoning \cite{marcus2020next}. Physical systems often involve causal relationships, temporal dependencies, and constraint satisfaction problems that may not align well with the associative reasoning patterns typical of LLMs.

The temporal dimension adds complexity to the grounding problem, as physical systems evolve continuously through both deterministic processes and stochastic events \cite{pearl2000causality}. LLMs must be able to reason about system dynamics, predict future states, and plan actions that account for temporal evolution and uncertainty.

Multimodal integration presents additional challenges, as physical systems generate data in various formats including numerical sensor readings, images, audio, and structured databases \cite{baltrusaitis2018multimodal}. Effective LLM integration requires approaches that can seamlessly combine these different data types while maintaining coherent reasoning capabilities.

Structured query integration represents a critical challenge for LLM-CPS applications, as physical systems often require precise queries to databases, simulation models, and control interfaces \cite{scholak2021duorat, yu2018spider}. LLMs must be able to generate accurate queries while handling complex schemas, relationships, and constraints typical of engineering systems.

Current approaches to structured query generation show promise in database applications but face challenges when extended to the complex, domain-specific interfaces typical of CPS \cite{li2023can, liu2023comprehensive}. These challenges include handling specialized data types, understanding complex relationships, and generating queries that respect system constraints and safety requirements.

The semantic gap between natural language reasoning and formal system interfaces presents ongoing challenges for LLM integration. Bridging this gap requires approaches that can translate high-level intentions into precise system commands while maintaining appropriate error handling and validation capabilities.

\section{Cognitive Agents: Integration Paradigms and Architectural Deficiencies}

The evolution from standalone language models to autonomous agents represents a significant advancement in AI capabilities, enabling systems that can interact with environments, use tools, and pursue complex objectives over extended time periods \cite{yao2022react, schick2023toolformer, qin2023tool}.

ReAct (Reasoning and Acting) represents an early paradigm for integrating LLM reasoning with external tool use \cite{yao2022react}. This approach interleaves reasoning steps with action execution, enabling more structured problem-solving approaches. However, ReAct was primarily designed for text-based environments and may not address the real-time constraints and safety requirements of physical systems.

Tool-using capabilities have been extensively developed for LLM agents, enabling integration with calculators, search engines, databases, and various APIs \cite{schick2023toolformer, nakano2021webgpt}. These developments demonstrate the potential for LLMs to interact with external systems, but current implementations focus primarily on information retrieval and simple computational tasks.

Multi-agent systems and collaborative reasoning approaches show promise for complex problem-solving tasks \cite{wu2023autogen, hong2023metagpt}. These systems can decompose complex problems, assign specialized roles to different agents, and coordinate activities toward common objectives. However, current implementations typically operate in digital environments and may not address the coordination challenges typical of physical systems.

Planning and reasoning frameworks have been developed to enable more sophisticated agent behavior, including hierarchical planning, goal decomposition, and constraint satisfaction \cite{huang2022language, ahn2022can}. These approaches demonstrate capabilities in task planning and execution but often assume static environments and may not handle the dynamic conditions typical of physical systems.

Digital domain successes demonstrate the potential of LLM agents in controlled environments such as software development, data analysis, and content creation \cite{nijkamp2022codegen, chen2021evaluating}. These applications benefit from well-defined interfaces, clear success criteria, and limited safety constraints that facilitate effective agent operation.

Code generation and software development represent particularly successful applications of LLM agents, where systems can understand requirements, generate solutions, and iterate based on feedback \cite{austin2021program, li2022competition}. These successes provide insights into effective agent architectures but operate in environments with very different characteristics from physical systems.

Web-based agents have demonstrated capabilities in navigation, information gathering, and task completion across various online platforms \cite{deng2023mind2web, zhou2023webarena}. While these applications involve some aspects of real-world interaction, they operate in highly structured digital environments that differ significantly from physical systems.

The cognitive-physical gap represents the fundamental challenge in extending agent capabilities from digital to physical domains \cite{marcus2020next, lake2017building}. This gap encompasses differences in environment dynamics, constraint types, safety requirements, and feedback mechanisms that require specialized approaches for effective bridging.

Physical systems involve continuous processes, real-time constraints, and irreversible actions that create fundamentally different operating conditions from digital environments. Agent architectures must account for these differences while maintaining effective reasoning and decision-making capabilities.

Safety and reliability requirements in physical systems often exceed those in digital applications, requiring approaches that can guarantee safe operation even under uncertain conditions \cite{leveson2011engineering}. Current agent architectures may not provide sufficient guarantees for safety-critical applications.

Real-time performance requirements create additional challenges, as physical systems often require responses within strict time limits that may conflict with the deliberative nature of LLM reasoning \cite{buttazzo2011hard}. Developing approaches that balance reasoning quality with response time represents an ongoing challenge.

Current integration approaches often adopt piecemeal solutions that address individual challenges without considering the systemic nature of physical system integration. These approaches may work for simple applications but are unlikely to scale to complex CPS that require comprehensive integration across multiple domains.

Evaluation frameworks for agent systems typically focus on task completion rates and accuracy measures that may not capture the full complexity of physical system interaction \cite{liu2023agentbench}. Developing appropriate evaluation approaches for physical domain agents remains an open challenge.

The lack of standardized benchmarks and evaluation environments limits the ability to compare different approaches and track progress in agent capabilities for physical applications. This limitation highlights the need for comprehensive evaluation frameworks that can assess agent performance across different types of physical tasks and environments.

