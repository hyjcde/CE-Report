% !TEX root = ../thesis.tex

\chapter{文献综述 (Literature Review)} \label{chp:literature}

\section{胃肠道肿瘤影像学诊断技术演进}

\subsection{从传统X线到现代断层成像}

胃肠道肿瘤的影像学诊断历史,是一部技术不断革新、诊断精度持续提升的历史。最早期的诊断依赖于上消化道钡餐造影等X线技术,这种方法通过观察造影剂的充盈缺损、粘膜破坏等间接征象来推断病变的存在,但其空间分辨率低,对早期平坦或微小病变的检出率极低,且受到操作者经验的极大影响。随着计算机断层扫描(CT)技术的出现,我们首次能够以断层的方式无重叠地观察腹部器官,极大地提升了对肿瘤大小、形态、以及与周围组织关系的评估能力,CT灌注成像等功能性技术也为肿瘤良恶性鉴别提供了更多信息。

紧随其后,磁共振成像(MRI),特别是弥散加权成像(DWI)等功能序列的应用,凭借其卓越的软组织分辨率和对细胞密度的敏感性,在肿瘤的定性诊断和分期方面展现出独特优势。而超声内镜(EUS)则将高频探头置于消化道管腔内,实现了对管壁层次和邻近器官的极致清晰显示,成为评估肿瘤浸润深度(T分期)和区域淋巴结状况(N分期)的金标准。然而,如前所述,这些技术的进步主要集中在"问题解决"而非"问题发现"阶段,其固有的成本、辐射或侵入性问题,使其难以承担起一线筛查的重任。

\subsection{超声技术的发展:从A型到实时三维成像}

超声医学作为一门独立的影像学科,其自身也经历了从一维到三维、从静态到动态、从形态到功能的巨大飞跃。最初的A型(振幅调制)超声只能提供一维的距离和回声强度信息。B型(亮度调制)超声的出现,首次将回声信号转化为二维的灰阶图像,奠定了现代超声诊断的基础。多普勒技术的引入,使得血流的探测成为可能,彩色多普勒和能量多普勒极大地丰富了对病灶血供情况的评估手段,这在肿瘤鉴别诊断中至关重要。

近年来,超声造影(CEUS)、弹性成像(Elastography)和实时三维/四维成像等新技术的涌现,进一步从血流灌注模式、组织硬度和空间形态等多个维度提升了超声的诊断能力。尽管技术日新月异,但所有这些先进的超声技术,都未能从根本上解决一个核心问题——它们依然是"工具",其价值的发挥,最终还是要通过操作者的"手"和"眼"来实现。这种对人的依赖性,是超声技术发展至今仍未被撼动的底层逻辑,也是本研究试图用AI来辅助和增强的根本出发点。

\section{人工智能在医学影像分析中的应用}

\subsection{机器学习与影像组学时代}

人工智能在医学影像分析中的应用可以追溯到20世纪80年代的早期专家系统,但真正的突破始于21世纪初机器学习技术的成熟。传统机器学习方法,如支持向量机(SVM)、随机森林等,需要研究者手工设计特征,这些方法在某些特定任务上取得了一定成功,但受限于特征工程的质量和数据的复杂性。影像组学(Radiomics)的概念在此背景下应运而生,它试图从医学影像中提取大量的定量特征,包括形状、纹理、密度等,并通过机器学习算法建立与临床终点的关联模型。

影像组学方法在多种癌症的诊断、分期和预后预测中显示出潜力,特别是在利用现有的CT、MRI等高分辨率图像方面。然而,影像组学的成功很大程度上依赖于精确的病灶分割和标准化的图像获取条件,这在超声影像中往往难以保证。超声图像的操作者依赖性、图像质量的变异性以及缺乏标准化的获取协议,使得传统影像组学方法在超声领域的应用受到了很大限制。

\subsection{深度学习革命:卷积神经网络(CNN)及其变体}

2012年AlexNet在ImageNet竞赛中的突破性表现,标志着深度学习时代的到来,也为医学影像分析带来了革命性的变化。卷积神经网络(CNN)能够自动学习层次化的特征表示,从低级的边缘、纹理到高级的语义概念,这种端到端的学习方式避免了手工特征设计的主观性和局限性。在医学影像领域,CNN及其变体(如ResNet、DenseNet、U-Net等)迅速在各种任务中取得了令人瞩目的成果,包括图像分类、目标检测、语义分割等。

在超声影像分析中,CNN展现出了处理复杂纹理和低对比度图像的独特优势。与传统方法相比,深度学习模型能够更好地适应超声图像的噪声、伪影和变异性,这为解决超声诊断中的挑战提供了新的工具。然而,CNN在超声领域的应用也面临着数据稀缺、标注困难、以及模型泛化性等问题,特别是对于胃癌和胰腺癌这类相对少见的疾病,高质量的标注数据更是稀缺。

\subsection{Transformer架构在视觉任务中的兴起}

2020年Vision Transformer(ViT)的提出,将在自然语言处理中取得巨大成功的Transformer架构引入到计算机视觉领域。与CNN的局部感受野不同,Transformer通过自注意力机制能够捕捉全局的上下文信息,这在医学影像分析中具有独特的价值,特别是在需要整合多个解剖区域信息的任务中。随后出现的各种混合架构,如ConvNext、MaxViT等,试图结合CNN的局部特征提取能力和Transformer的全局建模能力。

在超声影像分析中,Transformer架构的全局建模能力为解决器官间的解剖关系建模提供了新的可能。例如,在上腹部超声扫查中,胃、胰腺、肝脏等器官的相对位置和相互关系对于准确诊断至关重要,而传统CNN的局部处理方式可能难以有效捕捉这些长程依赖关系。本研究中提出的统一框架,正是希望利用Transformer的这种全局建模能力,实现对胃和胰腺的协同评估。

\section{深度学习在超声影像分析中的特定挑战与进展}

\subsection{超声图像的物理特性带来的挑战(低信噪比、伪影、各向异性)}

超声图像具有独特的物理特性,这些特性既是其优势所在,也是AI算法需要克服的主要挑战。首先,低信噪比是超声图像的固有特征,声波在人体组织中传播时会发生衰减、散射和吸收,导致图像中存在大量的斑点噪声(Speckle Noise),这种噪声不是简单的高斯噪声,而是与组织微结构相关的乘性噪声,传统的去噪方法往往效果有限。其次,超声成像过程中容易产生各种伪影,包括混响伪影、侧瓣伪影、多径伪影等,这些伪影可能被误认为病理结构,或者掩盖真实的病灶。

各向异性是超声图像的另一个重要特征,即图像的分辨率在不同方向上是不均匀的,轴向分辨率通常优于横向分辨率,这种特性要求AI算法在设计时必须考虑这种不对称性。此外,超声图像的对比度和亮度严重依赖于声束与组织界面的夹角、组织的声学特性以及系统的增益设置等因素,这种变异性使得模型的泛化变得困难。近年来,研究者们开始专门针对超声图像的这些特性开发深度学习算法,包括设计专门的网络架构、损失函数和数据增强策略。

\subsection{现有研究现状:在甲状腺、乳腺、肝脏等器官的应用}

目前,AI在超声影像分析中的应用主要集中在甲状腺、乳腺、肝脏等器官,这些领域已经取得了相对成熟的研究成果。在甲状腺超声AI方面,多项研究表明深度学习模型在甲状腺结节的良恶性鉴别方面能够达到甚至超越有经验医生的水平,一些商业化的产品已经开始在临床中应用。乳腺超声AI的研究同样活跃,特别是在乳腺肿块的检测和BI-RADS分类方面,AI系统显示出了良好的性能。肝脏超声AI主要聚焦于肝脏疾病的诊断,包括脂肪肝、肝纤维化、肝癌等的检测和评估。

这些成功的应用为超声AI的发展提供了宝贵的经验,但也揭示了当前研究的一些局限性。首先,大多数研究都是针对单一器官、单一疾病的专用模型,缺乏跨器官、跨疾病的泛化能力。其次,现有的研究主要关注静态图像的分析,而忽视了超声动态扫查的时序信息。再者,大多数模型只能提供诊断结果,而无法像有经验的医生那样提供诊断推理过程或置信度评估。最后,现有研究普遍缺乏与临床工作流的深度整合,大多停留在离线验证阶段,距离真正的临床应用还有很大距离。

\section{聚焦胃癌与胰腺癌的AI研究现状}

\subsection{基于EUS的AI研究}

超声内镜(EUS)作为诊断胃癌和胰腺癌的重要工具,近年来也成为AI研究的热点领域。基于EUS的AI研究主要集中在胰腺癌的早期诊断和T分期评估方面。多项研究表明,深度学习模型能够在EUS图像中准确识别胰腺癌,部分研究报告的AUC值超过0.9。在胃癌方面,基于EUS的AI研究相对较少,主要关注早期胃癌的检测和浸润深度的评估。一些研究尝试利用EUS-FNA(细针穿刺)的图像进行细胞学诊断的AI辅助。

然而,基于EUS的AI研究面临着数据获取困难、标注复杂、以及临床应用受限等问题。EUS作为一种侵入性检查,其数据相对稀缺,且需要高度专业化的医生进行操作和解读。此外,EUS图像的质量和视角高度依赖于操作者的技能,这增加了模型训练和验证的复杂性。最重要的是,EUS无法作为一线筛查工具,这限制了其在早期诊断中的应用价值。

\subsection{基于CT/MRI的AI研究}

基于CT/MRI的胃癌和胰腺癌AI研究相对成熟,这主要得益于这些影像技术的标准化程度较高和图像质量相对稳定。在胰腺癌方面,多项研究表明AI模型能够在CT图像中准确检测胰腺癌,并进行TNM分期评估。一些研究还尝试利用AI预测胰腺癌的可切除性和预后。在胃癌方面,基于CT的AI研究主要关注肿瘤的检测、分期和治疗反应评估。MRI在胃癌诊断中的应用相对较少,但在评估肿瘤浸润深度和淋巴结转移方面显示出潜力。

这些研究虽然在技术指标上取得了不错的成果,但在临床转化方面仍面临挑战。首先,CT/MRI的高成本和辐射风险使其无法用于大规模筛查。其次,大多数研究都是基于回顾性数据,缺乏前瞻性验证。再者,现有模型往往只关注技术性能指标,而忽视了临床决策的实际价值。最后,这些研究大多是单中心、小样本的,模型的泛化性有待验证。

\subsection{基于经腹超声的AI研究:仍处早期,成果零散}

相比于EUS和CT/MRI,基于经腹超声的胃癌和胰腺癌AI研究仍处于早期阶段,成果相对零散。现有的少数研究主要集中在胰腺癌的检测方面,利用经腹超声图像训练深度学习模型来识别胰腺肿块。一些研究尝试利用超声图像的纹理特征和形态学特征来鉴别胰腺癌的良恶性。在胃癌方面,基于经腹超声的AI研究更为稀少,主要原因是胃部超声检查在临床实践中相对较少,且技术难度较大。

这种研究现状的形成有多方面的原因。首先,经腹超声的操作者依赖性使得数据的标准化和质量控制变得困难。其次,胃癌和胰腺癌在经腹超声中的影像表现相对隐匿,增加了AI模型学习的难度。再者,现有的研究缺乏大规模、高质量的数据集支撑。最后,缺乏统一的评估标准和临床验证方法,使得不同研究之间难以比较和验证。

\subsection{知识空白识别 I:"单一任务,单一模型"范式的普遍性与局限性}

通过对现有文献的系统梳理,我们识别出第一个重要的知识空白:"单一任务,单一模型"范式的普遍性与局限性。目前绝大多数AI研究都采用这种范式,即为每个特定的诊断任务(如胃癌检测、胰腺癌分割、良恶性鉴别等)训练独立的专用模型。这种方法虽然在单一任务上可能取得不错的性能,但存在以下局限性:首先,它割裂了器官间的解剖关系和病理关联,无法利用胃和胰腺之间的协同信息。其次,这种方法导致了资源的重复投入和模型的碎片化,增加了临床部署的复杂性。

再者,单一任务模型往往缺乏上下文理解能力,无法像有经验的医生那样进行整体性的评估。最后,这种方法限制了知识的跨任务转移,新任务往往需要从头开始训练,效率低下。因此,迫切需要探索统一的、能够处理多器官、多任务的AI框架,这正是本研究的核心动机之一。

\section{统一模型与多任务学习的前沿探索}

\subsection{多任务学习(Multi-Task Learning)的理论基础与优势}

多任务学习(MTL)作为机器学习的重要分支,其核心思想是通过同时学习多个相关任务来提升单个任务的性能。MTL的理论基础可以追溯到人类学习的认知机制——人类在学习新技能时往往会利用已有的知识和经验,相关任务之间的知识迁移能够加速学习过程并提升最终的性能。在深度学习框架下,MTL通过共享网络的底层表征来实现知识的跨任务传递,这种共享机制使得模型能够学习到更加鲁棒和泛化的特征表示。

MTL的优势主要体现在以下几个方面:首先,数据效率的提升。通过利用多个任务的数据,MTL能够在单个任务数据稀缺的情况下仍然训练出有效的模型。其次,泛化能力的增强。共享表征的学习迫使模型关注任务间的共性特征,而非过度拟合特定任务的特异性特征,从而提升了模型的泛化能力。再者,计算资源的节约。相比于训练多个独立模型,MTL通过参数共享显著减少了模型的总参数量和计算开销。最后,正则化效应。多任务的联合训练相当于隐式的正则化,有助于防止过拟合。

\subsection{知识蒸馏、自监督学习等先进训练策略}

除了传统的多任务学习,近年来涌现出多种先进的训练策略,为构建更强大的统一模型提供了新的工具。知识蒸馏(Knowledge Distillation)通过让小模型(学生)学习大模型(教师)的知识,实现了模型压缩和知识传递的双重目标。在本研究的背景下,知识蒸馏可以用于将多个专家模型的知识整合到一个统一的框架中,或者用于将大规模预训练模型的知识迁移到特定的医学领域。

自监督学习(Self-Supervised Learning)通过设计巧妙的预训练任务,利用大规模无标注数据学习强大的特征表示。在医学影像领域,自监督学习特别有价值,因为获取高质量的医学影像标注往往成本高昂且耗时费力。对比学习、掩码图像建模等自监督方法在自然图像上取得了显著成功,并开始在医学影像领域显示出潜力。本研究提出的"两阶段知识注入"策略,正是借鉴了自监督学习的思想,首先利用大规模弱标注或无标注的腹部超声数据进行预训练。

\subsection{知识空白识别 II:缺乏针对超声异构数据特性设计的统一框架}

通过对多任务学习和统一模型文献的深入分析,我们识别出第二个重要的知识空白:现有的统一框架大多是针对自然图像或其他医学影像模态设计的,缺乏专门针对超声异构数据特性的设计考虑。超声图像具有独特的物理特性,包括各向异性的分辨率、复杂的噪声模式、以及高度的操作者依赖性等,这些特性要求AI框架在设计时必须进行特殊考虑。

现有的统一框架往往假设输入数据具有相对一致的质量和格式,这在超声领域显然不成立。超声数据的异构性不仅体现在图像质量的变异上,还体现在扫查视角、器官可视化程度、以及标注粒度等多个方面。例如,在上腹部超声扫查中,胃部的显示可能因为气体干扰而不完整,胰腺的可视化程度可能因为体型和肠气而变化很大,这种数据的异构性给统一建模带来了独特的挑战。因此,迫切需要开发专门针对超声数据特性的统一框架,这正是本研究的核心贡献之一。

\section{AI模型临床验证的方法学研究}

\subsection{从技术指标(AUC, Accuracy)到临床效用指标}

传统的AI模型评估主要依赖于技术性能指标,如准确率(Accuracy)、敏感性(Sensitivity)、特异性(Specificity)和受试者工作特征曲线下面积(AUC)等。这些指标虽然能够客观地衡量模型的分类性能,但往往无法直接反映模型在临床实践中的真实价值。临床决策是一个复杂的过程,涉及多种因素的综合考虑,包括诊断的准确性、治疗的获益风险比、患者的偏好、以及医疗资源的配置等。单纯的技术指标往往忽视了这些临床决策的复杂性。

近年来,医学AI领域开始重视临床效用指标的发展和应用。这些指标试图更直接地衡量AI模型对临床决策和患者结局的影响。例如,诊断影响(Diagnostic Impact)评估AI建议对最终诊断的影响程度;治疗变化(Treatment Change)衡量AI建议对治疗方案选择的影响;临床结局(Clinical Outcomes)直接评估AI应用对患者健康结局的改善程度。这种从技术指标向临床效用指标的转变,反映了医学AI研究从技术导向向临床价值导向的重要转变。

\subsection{决策曲线分析(DCA)的应用与价值}

决策曲线分析(Decision Curve Analysis, DCA)是近年来在医学预测模型评估中广泛应用的一种方法,它通过量化不同决策阈值下的净获益来评估模型的临床价值。DCA的核心思想是,在不同的风险阈值下,比较使用预测模型、不使用模型(默认治疗所有患者或都不治疗)等不同策略的净获益。净获益考虑了真阳性带来的益处和假阳性造成的损害,从而提供了更加贴近临床实际的模型评估视角。

在本研究的背景下,DCA特别有价值,因为胃癌和胰腺癌的诊断涉及重大的临床决策。早期诊断虽然能够显著改善患者预后,但过度诊断也可能导致不必要的焦虑、额外检查和治疗风险。DCA能够帮助我们量化在不同风险阈值下,使用AI辅助诊断相比于传统方法的净获益,从而为临床决策提供更有价值的信息。本研究计划在多维度验证体系中纳入DCA分析,以全面评估所提出统一框架的临床价值。

\subsection{知识空白识别 III:现有研究普遍缺乏与病理金标准和临床决策的深度链接}

通过对AI模型临床验证方法学的系统回顾,我们识别出第三个重要的知识空白:现有的超声AI研究普遍缺乏与病理金标准和临床决策的深度链接。大多数研究的验证仍然停留在影像标注的层面,即比较AI预测与影像医生标注的一致性,而很少将AI预测与最终的病理诊断或临床结局进行直接比较。这种验证方式存在两个主要问题:首先,影像医生的标注本身可能存在误差,将其作为"金标准"可能低估或高估AI的真实性能。

其次,影像诊断的准确性并不等同于临床价值。一个在影像标注上表现优秀的AI模型,未必能在实际临床决策中提供有价值的帮助。病理诊断作为肿瘤诊断的最终金标准,应该成为AI模型验证的重要参考。此外,现有研究往往忽视了AI预测对临床决策路径的影响,缺乏对AI辅助诊断在改善患者管理、优化资源配置等方面价值的系统评估。本研究将建立"技术-临床-决策"三位一体的验证体系,正是为了填补这一重要的知识空白。

\section{医疗领域的人机交互与工作流整合}

\subsection{临床决策支持系统(CDSS)的设计原则}

临床决策支持系统(Clinical Decision Support Systems, CDSS)作为医学AI的重要应用形式,其设计原则直接影响着AI技术在临床实践中的接受度和有效性。成功的CDSS通常遵循几个核心设计原则:首先是非干扰性(Non-intrusiveness),即系统应该无缝地整合到现有的临床工作流中,而不是强制医生改变既有的工作习惯。过于突兀或复杂的界面往往会引起医生的抵触情绪,影响系统的采用。其次是可解释性(Explainability),医生需要理解AI建议的基础和逻辑,盲目的"黑盒"建议往往难以获得临床医生的信任。

再者是上下文感知性(Context-awareness),优秀的CDSS应该能够理解当前的临床情境,包括患者的具体情况、检查的目的、以及医生的专业背景等,从而提供个性化的建议。最后是渐进式的智能增强(Progressive Intelligence Augmentation),系统应该根据医生的经验水平和偏好,提供不同程度的辅助,从简单的提醒到复杂的决策建议。本研究提出的Sono-Agent原型正是基于这些设计原则,旨在构建一个真正有用和受欢迎的临床AI助手。

\subsection{生成式AI在医学报告生成中的机遇与风险("AI幻觉")}

大语言模型和生成式AI的快速发展为医学报告的自动生成提供了新的可能性。这些技术能够根据影像发现、临床信息等输入,生成流畅、专业的医学报告文本,显著提升报告撰写的效率。在超声领域,报告生成AI的价值尤为突出,因为超声检查往往涉及大量的实时观察和主观判断,将这些信息转化为标准化的报告文本是一个耗时且容易出错的过程。

然而,生成式AI在医学报告生成中也面临着独特的风险,其中最突出的是"AI幻觉"(AI Hallucination)问题。AI幻觉指的是模型生成看似合理但实际上不准确或不基于真实证据的内容。在医学领域,这种幻觉可能导致严重的后果,包括误诊、不当治疗或医疗纠纷。为了解决这一问题,研究者们提出了多种策略,包括基于检索的生成(Retrieval-Augmented Generation)、事实核查机制、以及人机协作的报告生成等。本研究在Sono-Agent原型中特别关注这一问题,提出了知识增强的智能报告生成方法,通过引入医学知识图谱进行事实核查,确保生成内容的可靠性。

\subsection{知识空白识别 IV:缺乏专为超声动态扫查设计的协同智能交互范式}

通过对人机交互和工作流整合文献的深入分析,我们识别出第四个重要的知识空白:现有的医学AI交互系统大多是针对静态影像和标准化工作流设计的,缺乏专为超声动态扫查设计的协同智能交互范式。超声检查具有高度动态和交互性的特点,医生在扫查过程中需要不断调整探头位置、改变扫查角度、并根据实时观察到的情况决定下一步的检查策略。这种动态的、适应性的检查过程与静态影像的分析有着本质的不同。

现有的AI系统往往假设输入是预先确定的、标准化的图像,而忽视了超声检查中的这种动态性和交互性。此外,超声医生在扫查过程中往往依赖于多种感官信息的整合,包括视觉、触觉反馈等,这种多模态的信息整合也是现有AI系统难以处理的。因此,迫切需要开发专门针对超声动态扫查特点的协同智能交互范式,这种范式应该能够在检查过程中提供实时、上下文相关的智能辅助,同时保持对医生决策自主权的尊重。本研究提出的Sono-Agent正是为了填补这一空白。

\section{本研究的定位与出发点总结}

通过以上的系统文献综述,我们清晰地识别出了四个重要的知识空白,这些空白共同构成了本研究的理论基础和实践动机。首先,"单一任务,单一模型"范式的普遍性与局限性揭示了构建统一计算框架的必要性。现有的AI研究缺乏对胃癌和胰腺癌之间协同效应的利用,这不仅限制了诊断性能的提升,也增加了临床部署的复杂性。本研究提出的USANet统一框架,正是为了突破这一范式的局限,实现真正的协同智能。

其次,缺乏针对超声异构数据特性设计的统一框架暴露了现有方法在处理超声数据时的不足。超声图像的独特物理特性和高度变异性要求AI框架进行专门的设计考虑,而现有的通用框架往往无法有效处理这些挑战。本研究提出的"两阶段知识注入"训练策略,特别考虑了超声数据的这些特殊性质。

再者,现有研究普遍缺乏与病理金标准和临床决策的深度链接,限制了AI模型临床价值的准确评估。本研究建立的"技术-临床-决策"三位一体验证体系,旨在更全面、更准确地评估AI框架的临床价值,特别是通过决策曲线分析量化其在临床决策中的净获益。

最后,缺乏专为超声动态扫查设计的协同智能交互范式阻碍了AI技术在超声领域的深度应用。本研究提出的Sono-Agent原型,试图构建一种新的人机协同工作流,这种工作流既能充分利用AI的技术优势,又能保持和增强医生的专业判断能力。

综合而言,本研究定位于医学AI、超声诊断和临床决策支持的交叉领域,旨在通过系统性的理论创新和技术突破,为解决胃癌和胰腺癌早期诊断这一重大临床挑战提供完整的解决方案。研究的出发点不仅是技术的先进性,更是临床的实用性和转化价值,力求在推动学术前沿的同时,真正造福患者和社会。

