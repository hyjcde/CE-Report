% !TEX root = ../thesis.tex

\chapter{Introduction} \label{chp:intro}

\section{Background and Motivation}

The advent of Large Language Models (LLMs) has marked a transformative milestone in artificial intelligence, demonstrating unprecedented capabilities in natural language understanding, reasoning, and generation. These models have shown remarkable proficiency in tasks ranging from text completion and translation to complex problem-solving and creative writing. However, as we seek to deploy LLMs in real-world applications that require interaction with dynamic physical environments, a critical limitation emerges: the fundamental disconnect between their abstract reasoning capabilities and the concrete realities of the physical world.

LLMs derive their knowledge from vast corpora of static text data, resulting in internal world models that are inherently disembodied and lack temporal context. This limitation becomes particularly pronounced when these models are tasked with making decisions or generating plans that must account for physical constraints, spatial relationships, temporal dynamics, and real-time environmental states. The consequences of this disconnect can range from suboptimal performance to potentially dangerous outcomes in safety-critical applications.

\section{Problem Statement}

The core challenge addressed in this thesis is the development of a systematic approach to bridge the gap between LLM reasoning capabilities and physical world understanding. Specifically, we identify the need for:

\begin{itemize}
    \item A dynamic and contextual world representation that can be actively queried and manipulated by LLMs
    \item A structured decision-making framework that grounds abstract reasoning in physical reality
    \item A feedback mechanism that enables continuous learning and adaptation based on real-world interactions
    \item Validation across diverse domains to demonstrate the generalizability of the approach
\end{itemize}

\section{Proposed Solution: The CORTEX Architecture}

To address these challenges, this thesis introduces CORTEX (Cognitive Reasoning and Task EXecution architecture), a novel cognitive architecture that fundamentally reframes the role of LLMs in physical world interaction. Rather than treating LLMs as standalone reasoning engines, CORTEX positions them as cognitive cores that are intrinsically dependent on dynamic world representations for effective decision-making.

The key innovation of CORTEX lies in its integration of Digital Twins (DTs) as the primary vehicle for world representation. We define a Digital Twin as any computational model that dynamically represents a physical system with sufficient fidelity to support task-specific decision-making requirements. This broad definition encompasses various forms of representation, from high-fidelity 3D geometric models to abstract feature-space representations.

\section{Contributions}

The primary contributions of this thesis are:

\begin{enumerate}
    \item \textbf{Architectural Innovation}: The development of the CORTEX cognitive architecture, which provides a systematic framework for LLM-driven decision-making in physical environments.
    
    \item \textbf{Theoretical Foundation}: A comprehensive analysis of the limitations of current LLM approaches in physical world interaction and the theoretical basis for Digital Twin-enhanced reasoning.
    
    \item \textbf{Implementation Framework}: A practical four-stage cognitive loop that operationalizes the integration of LLMs with Digital Twin environments.
    
    \item \textbf{Cross-Domain Validation}: Empirical validation through three diverse case studies spanning building health monitoring, medical diagnosis, and autonomous exploration.
    
    \item \textbf{Performance Enhancement}: Demonstrated improvements in decision quality, robustness, and safety across all evaluated domains.
\end{enumerate}

\section{Structure of the Thesis}

This thesis is organized as follows: \autoref{chp:preliminaries} provides the necessary background on LLMs, Digital Twins, and cognitive architectures. \autoref{chp:proof} presents the detailed design and implementation of the CORTEX architecture. \autoref{chp:applications} demonstrates the effectiveness of the proposed approach through comprehensive case studies across three distinct domains. Finally, the thesis concludes with a discussion of implications, limitations, and future research directions.