% !TEX root = ../thesis.tex

\chapter{Introduction} \label{chp:intro}

\section{Gastric and Pancreatic Cancers: A Critical Modern Medical Challenge}

\subsection{Alarming Mortality Rates and Stagnant Survival Outcomes}

Gastric and pancreatic cancers represent two of the most lethal malignancies within the digestive system, whose threat to global public health is disproportionate to their incidence rates. According to the latest global cancer statistics, these two cancer types exhibit an exceptionally high mortality-to-incidence ratio, indicating that once diagnosed, patients typically face extremely poor prognoses. This grim reality is reflected in five-year survival rates that have remained persistently low despite decades of medical advancement. While late-stage treatment modalities such as targeted therapy and immunotherapy have achieved certain progress, the overall improvement in five-year survival rates has been minimal, with gastric cancer remaining below 35\% and pancreatic cancer falling short of 12\%, ranking among the lowest across all major cancer types.

These stagnant survival statistics send a clear signal to the entire medical community that relying solely on treatment strategies for advanced disease has reached a ceiling. The stark contrast between high treatment costs and minimal survival benefits places tremendous burden on patient families and society as a whole. Consequently, the strategic shift of research and clinical practice focus from "treating advanced disease" to "detecting early disease" has become an urgent consensus and core mission within the oncology community. This paradigm shift represents not merely an incremental improvement but a fundamental reconceptualization of cancer care delivery.

\subsection{Early Detection: The Singular Key to Changing Outcomes}

The medical community universally acknowledges that the sole and most critical pathway to significantly improving survival rates for gastric and pancreatic cancer patients lies in achieving genuine early detection. When tumors remain confined to their organ of origin without lymph node metastasis or distant spread, the possibility of curative surgical resection is maximized, and patient prognosis is most favorable. A compelling example illustrates this principle: early-stage gastric cancer, when discovered timely and treated with endoscopic resection or radical surgery, can achieve five-year survival rates exceeding 90\%, creating a stark contrast with advanced gastric cancer survival rates below 10\%.

However, the early symptoms of these two cancer types are often extremely subtle and non-specific, frequently mistaken for common gastric discomfort or indigestion. Patients typically seek medical attention only when significant symptoms appear, such as jaundice, severe abdominal pain, or rapid weight loss, by which time the disease has often progressed to intermediate or advanced stages. This "silent killer" characteristic renders symptom-driven diagnostic approaches completely ineffective, creating an urgent need for technologies capable of effective screening in large-scale asymptomatic or high-risk populations. The challenge extends beyond mere technical capability to encompass accessibility, cost-effectiveness, and implementation feasibility across diverse healthcare settings.

\section{The "Screening Gap" in Current Diagnostic Technologies and Ultrasound Opportunities}

\subsection{Limitations of CT, MRI, and EUS in First-Line Screening}

Within the current medical imaging arsenal, computed tomography (CT), magnetic resonance imaging (MRI), and endoscopic ultrasound (EUS) undoubtedly represent powerful tools for diagnosing gastrointestinal tumors. These technologies play indispensable roles in disease confirmation, staging, and treatment planning. CT excels with rapid imaging and global visualization capabilities, MRI provides unparalleled soft tissue contrast, and EUS delivers exceptional near-field high-resolution imaging capabilities that have established it as the gold standard for assessing tumor invasion depth and regional lymph node status.

Nevertheless, the formidable capabilities of these technologies come hand-in-hand with substantial costs, invasiveness, or radiation risks, rendering them fundamentally unsuitable for large-scale first-line screening applications. The ionizing radiation associated with CT precludes its use for periodic screening of general populations. MRI examinations are prohibitively expensive, time-consuming, and require sophisticated equipment, limiting accessibility. EUS, as an invasive procedure, not only causes patient discomfort and risk but also demands complex operation protocols that preclude its use as a population screening tool. These inherent limitations collectively create a massive "screening gap" at the beginning of the diagnostic pathway, representing our lack of a safe, economical, convenient, and effective tool to serve as the first line of defense.

This screening gap represents more than a mere technical challenge; it embodies a fundamental healthcare delivery problem that perpetuates late-stage diagnosis and poor outcomes. The economic burden of advanced cancer treatment far exceeds the cost of early detection programs, yet the absence of suitable screening technologies creates a perverse incentive structure that favors expensive late-stage interventions over preventive measures.

\subsection{Technical Advantages and Strategic Value of Transabdominal Ultrasound}

Transabdominal ultrasound (TAUS) emerges as the most ideal candidate for filling the aforementioned "screening gap" through its unique combination of attributes. Firstly, it is completely non-invasive and free from ionizing radiation, possessing extremely high safety profiles suitable for repeated examinations across any population demographic. Secondly, its equipment is relatively inexpensive and portable, resulting in extremely low examination costs and unparalleled accessibility and cost-effectiveness. Furthermore, TAUS enables real-time dynamic assessment, allowing observation of organ peristalsis, hemodynamic information, and other functional parameters that static imaging techniques cannot provide. Finally, it can perform rapid, comprehensive anatomical coverage of the entire abdominal cavity.

The synthesis of these technical advantages endows TAUS with tremendous strategic value, positioning it as a potential "fifth vital sign" following blood pressure and body temperature, serving as a fundamental tool for health examinations and opportunistic screening in high-risk populations. If the diagnostic efficacy of TAUS could be stabilized and enhanced, it would possess the capability to significantly advance the diagnostic timeline for tumors, fundamentally transforming the prevention and treatment landscape for gastric and pancreatic cancers. This transformation would represent a paradigm shift from reactive treatment to proactive prevention, aligning with contemporary healthcare goals of value-based care and population health management.

\section{The Transabdominal Ultrasound Paradox: Potential Versus Bottlenecks}

\subsection{Profound Operator Dependence and Skill Requirements}

As revealed by this research, transabdominal ultrasound presents a profound and fundamental paradox: an imaging modality with the highest screening potential is constrained by the lowest diagnostic consistency. On one hand, its safety, non-invasiveness, accessibility, cost-effectiveness, real-time dynamic assessment capabilities, and broad anatomical coverage make it theoretically the perfect screening tool. These advantages are unmatched by any expensive imaging technology, constituting the source of its tremendous potential.

On the other hand, this enormous potential is tightly constrained in clinical practice by a series of inherent limitations. These limitations encompass extreme dependence on operator skills and experience, susceptibility to interference from intestinal gas and other factors leading to poor visualization, extremely low contrast of early micro or isoechoic lesions making them difficult to detect, and ultimately resulting in significant diagnostic variability between different physicians and institutions. This operator dependence extends beyond mere technical proficiency to encompass years of accumulated experience, pattern recognition capabilities, and intuitive understanding of anatomical variations.

\subsection{Diagnostic Quality Variability: The Core Obstacle}

This paradox represents the core bottleneck currently preventing TAUS from becoming a reliable screening tool. The fundamental question arising from this paradox naturally leads to the core research question of this doctoral thesis: How can we systematically mitigate these inherent limitations to unlock the full potential of transabdominal ultrasound? Traditional approaches relying on extended physician training periods, while effective, are inefficient and far from meeting the growing clinical demand. Therefore, a technical, scalable solution must be sought.

This research proposes that artificial intelligence, particularly deep learning, provides an unprecedented historical opportunity to crack this paradox. AI has the potential to transform the tacit knowledge and diagnostic patterns of top experts into computable, reproducible explicit capabilities, thereby reducing dependence on individual operator experience, enhancing robustness under artifact interference, and augmenting sensitivity to subtle pathological signals. Consequently, the starting point of this research is to use AI as a key to unlock the tremendous potential of TAUS that has been locked by "operator dependence." This approach represents a convergence of technological capability and clinical need that was previously impossible.

\section{Research Philosophy Transformation: From "Analytical Tool" to "Cognitive Partner"}

\subsection{Synergistic Intelligence as Core Research Philosophy}

In seeking AI solutions, a key insight emerges: the stomach and pancreas possess inseparable intrinsic connections in anatomy, pathology, and clinical workflow. This research summarizes these as the "Four Pillars of Synergy." The first pillar, anatomical adjacency, recognizes that the stomach, particularly the antrum and body, directly overlies the pancreas anteriorly. When the gastric cavity is filled with fluid, it can form an excellent acoustic window free from intestinal gas interference, providing an opportunity to observe the pancreas that would otherwise be obscured. The second pillar, pathological interdependence, acknowledges that diseases of these two organs can mutually influence each other. For example, pancreatic tumors may invade or compress the gastric wall causing obstruction, while certain gastric lesions such as large ulcers may mimic pancreatic disease on imaging.

The third pillar encompasses integrated clinical workflow, recognizing that in standard upper abdominal ultrasound examination, physicians naturally perform systematic exploration of multiple organs including liver, gallbladder, pancreas, spleen, and stomach. This means that data acquisition for both stomach and pancreas is naturally integrated without requiring additional operations, providing an efficient data foundation for constructing unified models. The fourth pillar represents the highest level of synergistic effects through contextual AI analysis potential. A sufficiently intelligent AI model can utilize the status of one organ as contextual clues for analyzing another organ, creating emergent diagnostic capabilities that exceed the sum of individual components.

\subsection{Empowerment Rather Than Replacement: The Ultimate Goal of Human-AI Collaboration}

The theory of "Four Pillars of Synergy" directly leads to the core technical pathway of this research: constructing a unified computational framework. Traditional "single-task, single-model" research paradigms, such as one model dedicated solely to pancreatic segmentation and another exclusively for gastric cancer classification, fundamentally sever the intrinsic connections between these organs and cannot utilize the aforementioned synergistic effects. This fragmented approach cannot achieve contextual AI analysis nor simulate the holistic thinking present in real clinical workflows.

Therefore, the unified framework proposed by this research aims to design a single, end-to-end deep learning model capable of simultaneously receiving upper abdominal ultrasound image inputs and jointly, synchronously outputting multiple assessment results for both stomach and pancreas. This design enables the model to inherently capture and utilize anatomical relationships, pathological associations, and image contextual information between the two major organs during the learning process, potentially achieving or even exceeding the performance of specialized single-task models on individual organ assessment tasks. This unified framework serves as the technical carrier for realizing the "synergistic intelligence" concept, representing a fundamental shift from isolated diagnostic tools toward integrated cognitive systems.

\section{Core Scientific Questions}

Based on the background analysis, problem identification, and theoretical construction presented above, this doctoral thesis aims to systematically address three interconnected core scientific questions that form the foundation of comprehensive solution development.

\subsection{Research Question 1: The Construction Problem of Unified Computational Framework}

How can we construct a unified computational framework that effectively leverages the synergistic effects between stomach and pancreas? This extends far beyond a simple model concatenation problem, involving the design of advanced network architectures capable of simultaneously processing two organs and multiple tasks including detection, segmentation, classification, and attribute assessment. The challenge encompasses developing novel training strategies that can efficiently learn from heterogeneous data containing stomach, pancreas, or both organs while genuinely utilizing contextual information to enhance performance.

The complexity of this question lies in balancing multiple competing objectives while maintaining computational efficiency and clinical interpretability. The framework must handle varying image quality, different scanning protocols, and diverse pathological presentations while providing consistent, reliable outputs across different clinical scenarios. Furthermore, the architecture must be sufficiently flexible to accommodate future extensions and modifications as clinical understanding evolves and new diagnostic requirements emerge.

\subsection{Research Question 2: The Clinical Trust and Value Validation Problem}

How can we scientifically and rigorously validate the clinical value of this framework to establish clinical trust? This requires transcending traditional validation methods that focus solely on technical metrics such as area under the curve (AUC). We need to establish a multi-dimensional validation system that directly links AI model outputs with the most authoritative clinical endpoints, specifically pathological gold standards and postoperative TNM staging. Additionally, decision curve analysis and other tools must be employed to quantify genuine net benefit from a clinical decision-making perspective, answering the fundamental question: "Can this model help physicians make better decisions?"

The validation challenge extends beyond technical performance to encompass clinical workflow integration, user acceptance, and long-term impact on patient outcomes. The framework must demonstrate not only statistical significance but also clinical meaningfulness, cost-effectiveness, and practical implementability across diverse healthcare settings. This requires sophisticated study designs that can isolate the contribution of AI assistance while controlling for confounding variables and ensuring generalizability across different populations and practice environments.

\subsection{Research Question 3: The Seamless Clinical Workflow Integration Problem}

How can we seamlessly and safely integrate validated AI capabilities into real clinical workflows to achieve genuine human-AI collaboration? A significant gap exists between offline algorithmic models and clinically deployable products. This question focuses on "last mile" translation, specifically designing interaction prototypes that can provide intelligent, non-intrusive assistance during dynamic ultrasound scanning in real-time and efficiently, reliably generate report drafts after examination completion, thereby truly empowering physicians and enhancing their work efficiency and diagnostic confidence.

The integration challenge encompasses technical considerations such as real-time processing requirements, user interface design, and system reliability, as well as human factors including cognitive load, workflow disruption, and trust calibration. The solution must accommodate varying levels of user expertise, different clinical contexts, and diverse institutional requirements while maintaining safety standards and regulatory compliance. Success requires not only technological innovation but also deep understanding of clinical practice patterns and effective change management strategies.

\section{Research Objectives and Core Contributions}

To address the aforementioned scientific questions, this research establishes the following specific objectives that collectively form a comprehensive approach to advancing transabdominal ultrasound diagnostics through artificial intelligence.

The first objective involves designing and implementing a unified deep learning framework named USANet (Unified Sonographic Assessment Network) coupled with proposing an innovative "Two-Stage Knowledge Injection" training strategy. This framework represents a fundamental departure from existing single-task approaches by enabling simultaneous learning and execution of multiple assessment tasks for both gastric and pancreatic cancers within a single, coherent model architecture. The two-stage training strategy mimics human medical education by first learning general anatomical knowledge from large-scale, weakly-labeled abdominal ultrasound data, then specializing in disease-specific diagnostic knowledge through fine-tuning on precisely annotated cancer datasets.

The second objective encompasses constructing a multi-center, high-quality, precisely annotated transabdominal ultrasound dataset containing both gastric and pancreatic cancers, serving as the foundation for model training and validation. This dataset will be unprecedented in its scope, combining data from multiple institutions to ensure diversity and representativeness while maintaining rigorous quality control standards. The dataset construction process will establish new benchmarks for medical imaging dataset development, including innovative annotation protocols, quality assurance mechanisms, and ethical compliance frameworks.

The third objective involves establishing and executing a "technical-clinical-decision" tripartite comprehensive validation scheme to thoroughly evaluate USANet's performance and clinical value. This validation approach transcends traditional technical metrics to encompass clinical endpoint correlation and decision-making value assessment, providing a holistic understanding of the framework's potential impact on patient care. The validation scheme will serve as a model for future AI system evaluation in clinical settings, addressing current gaps in translation from technical performance to clinical utility.

The fourth objective entails designing and developing a human-AI collaborative workflow prototype named Sono-Agent, followed by preliminary evaluation through usability studies. This prototype represents the culmination of technical development efforts, providing a tangible demonstration of how advanced AI capabilities can be seamlessly integrated into clinical practice. The usability studies will provide crucial insights into real-world implementation challenges and opportunities for optimization.

The core contribution of this research lies in its systematic and innovative nature, spanning theoretical, technical, methodological, and application domains. It aims to provide a complete solution for addressing core bottlenecks in TAUS applications for early diagnosis of gastric and pancreatic cancers, encompassing theoretical foundations, technical implementation, clinical validation, and application integration. This comprehensive approach distinguishes the work from previous efforts that typically address isolated aspects of the challenge without considering the full spectrum of requirements for clinical translation and adoption.

\section{Thesis Structure Organization}

The structure of this thesis is organized around the three core scientific questions presented above, ensuring logical progression from problem identification through solution development to validation and application. The first chapter serves as the introduction, establishing the clinical motivation, technical challenges, and research objectives that guide the entire investigation.

The second chapter provides a comprehensive literature review that systematically examines the current state of related fields and further clarifies the knowledge gaps and positioning of this research. This review encompasses medical imaging technology evolution, artificial intelligence applications in medical image analysis, ultrasound-specific challenges and opportunities, and human-AI collaboration paradigms. The literature review serves not only to establish the research context but also to identify specific areas where this work makes novel contributions.

The third chapter details the overall design and core methodology of this research, with particular emphasis on dataset construction, USANet framework design and implementation, multi-dimensional validation system establishment, and Sono-Agent prototype design concepts. This chapter serves as the technical foundation for the entire thesis, providing sufficient detail for reproducibility while maintaining clarity for readers from diverse backgrounds.

The fourth chapter presents and analyzes experimental results from the USANet framework, addressing the first and second scientific questions through comprehensive technical validation and clinical correlation analysis. This chapter demonstrates the effectiveness of the unified approach compared to traditional single-task models and establishes the clinical relevance of the technical achievements.

The fifth chapter focuses on Sono-Agent prototype implementation and usability evaluation, addressing the third scientific question through real-world testing with clinical users. This chapter bridges the gap between technical capability and practical implementation, providing insights into the human factors and workflow considerations essential for successful clinical adoption.

The sixth chapter provides in-depth discussion of all research work, analyzing advantages and limitations while exploring future research directions. This discussion synthesizes findings across all three research questions and places them in the broader context of medical AI development and clinical practice evolution.

The seventh chapter presents comprehensive conclusions, reiterating core findings and contributions while outlining implications for future research and clinical practice. This final chapter serves to consolidate the research achievements and provide clear guidance for continuation and extension of this work by future investigators.
