% !TEX root = ../thesis.tex

\chapter{The CORTEX Cognitive Architecture: Design and Implementation} \label{chp:cortex}

% Chapter 3 Outline:
% 3.1 Architectural Overview and Design Principles
% 3.2 Core Component I: Task-Specific Digital Twin Framework
% 3.3 Core Component II: Four-Stage Cognitive Loop
% 3.4 System Integration and Implementation Architecture
% 3.5 Theoretical Analysis and Properties

\section{Architectural Overview and Design Principles}

\subsection{Design Philosophy}
% - LLM as cognitive core dependent on world representation
% - Digital Twin as bridge between symbolic and physical
% - Continuous learning and adaptation paradigm
% - Safety-first design principles

\subsection{System Requirements and Constraints}
% - Real-time performance requirements
% - Scalability across different domains
% - Safety and reliability constraints
% - Interpretability and explainability needs

\subsection{Comparison with Existing Approaches}
% - Traditional cognitive architectures (SOAR, ACT-R)
% - Modern LLM-based agent frameworks
% - Digital Twin applications in engineering
% - Unique advantages of CORTEX integration

\section{Core Component I: Task-Specific Digital Twin Framework}

\subsection{Functional Definition of Digital Twins}
% - Computational model with sufficient fidelity
% - Task-specific adaptation and optimization
% - Multi-modal data integration capabilities
% - Real-time update and calibration mechanisms

\subsection{Digital Twin Design Patterns}
% - High-fidelity 3D geometric models (UAV exploration)
% - Multi-dimensional feature spaces (medical diagnosis)
% - Temporal-spatial fusion models (building monitoring)
% - Abstract symbolic representations

\subsection{Data Integration and Processing Pipeline}
% - Sensor data acquisition and preprocessing
% - Model update and synchronization protocols
% - Quality assurance and validation procedures
% - Performance monitoring and optimization

\section{Core Component II: Four-Stage Cognitive Loop}

\subsection{Stage 1: Perceptual Grounding \& Context Formulation}
% - Digital Twin querying mechanisms
% - Context extraction and representation
% - Attention and focus management
% - Multi-modal information fusion

\subsection{Stage 2: Causal Inference \& Predictive Simulation}
% - Causal reasoning within Digital Twin environment
% - What-if scenario generation and analysis
% - Uncertainty quantification and propagation
% - Temporal prediction and forecasting

\subsection{Stage 3: Action Policy Generation \& Validation}
% - LLM-based policy generation
% - Safety constraint checking
% - Policy validation through simulation
% - Risk assessment and mitigation strategies

\subsection{Stage 4: Physical Interaction \& Model Calibration}
% - Action execution in physical world
% - Feedback collection and processing
% - Model update and recalibration
% - Learning and adaptation mechanisms

\section{System Integration and Implementation Architecture}

\subsection{Software Architecture and Components}
% - Modular design and component interfaces
% - Communication protocols and data flow
% - Error handling and fault tolerance
% - Performance monitoring and optimization

\subsection{Hardware Requirements and Deployment}
% - Computational resource requirements
% - Network and communication infrastructure
% - Sensor integration and interfacing
% - Edge computing and distributed processing

\subsection{Implementation Technologies and Frameworks}
% - LLM integration and API management
% - Digital Twin simulation engines
% - Real-time data processing systems
% - Visualization and monitoring tools

\section{Theoretical Analysis and Properties}

\subsection{Convergence and Stability Analysis}
% - Cognitive loop convergence properties
% - System stability under uncertainties
% - Robustness to model inaccuracies
% - Performance bounds and guarantees

\subsection{Computational Complexity and Scalability}
% - Time and space complexity analysis
% - Scalability with system size and complexity
% - Resource allocation and optimization
% - Trade-offs between accuracy and efficiency

\subsection{Safety and Reliability Properties}
% - Formal verification approaches
% - Safety constraint satisfaction
% - Fault detection and recovery mechanisms
% - Graceful degradation strategies

\subsection{Chapter Summary}
% - Key architectural innovations
% - Implementation considerations
% - Theoretical foundations and properties
% - Bridge to empirical validation chapters

% TODO: Add detailed technical content for each section
% Current status: Outline completed, implementation details to be added
% Target completion: Before Candidacy Examination (July 2025)