% !TEX root = ../thesis.tex

胃癌与胰腺癌因其高致死率和低早期诊断率,构成了全球性的重大健康挑战,迫切需要创新的诊断技术方法。尽管经过数十年的医学进步,这些致命疾病的五年生存率仍然顽固地保持在较低水平,胃癌生存率低于35\%,胰腺癌生存率不足12\%,这主要归因于早期疾病的隐匿性特征,往往在晚期阶段才被发现,此时治疗选择已经变得有限。根本挑战不在于治疗晚期疾病(尽管采用了复杂的干预措施,但已达到治疗上限),而在于实现真正的早期检测,此时治愈性干预仍然可行,患者预后最为有利。

经腹超声检查(TAUS)凭借其安全无创、便捷经济、实时动态评估及宽泛解剖覆盖的显著优势,成为填补一线诊断技术中关键筛查空白的最有希望的候选者。然而,TAUS存在一个限制其临床应用的根本性悖论:具有最高筛查潜力的成像技术却受到最低诊断一致性的制约。这种诊断悖论的特征是极度依赖操作者经验以及对伪影干扰下早期病灶的低敏感性,严重限制了其巨大潜力的完全实现。推动本研究的核心问题是:我们如何能够通过创新的技术方法系统性地缓解这些固有局限性,从而释放TAUS的全部潜力?

本论文通过"协同智能"的理论框架来应对这一挑战,认识到胃癌和胰腺癌在解剖学、病理学和临床工作流程中具有不可分割的内在联系,这些联系可以通过统一的计算方法加以利用。传统的"单任务、单模型"研究范式人为地分割了这些自然关系,错失了进行上下文分析和跨器官信息利用的机会。本研究提出,人工智能,特别是通过统一的多任务学习架构,为捕获和利用这些协同效应提供了前所未有的机会,同时减少对个体操作者经验的依赖。

本研究通过全面的实验调查和临床验证,系统地解决了三个相互关联的科学问题。首先,我们如何构建一个能够有效利用胃和胰腺之间协同效应的统一计算框架?这个问题促成了USANet(统一超声评估网络)的开发,这是一种采用CNN-Transformer混合设计的新颖深度学习架构,结合了创新的"两阶段知识注入"训练策略。第一阶段利用大规模无标注腹部超声数据通过自监督学习方法进行通用解剖表征学习,第二阶段使用精确标注的胃癌和胰腺癌数据集进行多任务知识联合微调。

其次,我们如何科学严谨地验证该框架的临床价值以建立临床信任?这个问题促使建立了一个全面的"技术-临床-决策"三位一体验证系统,该系统超越了传统的技术指标,涵盖了临床终点关联和决策价值评估。验证框架直接将AI模型输出与病理金标准联系起来,并采用决策曲线分析从临床决策制定的角度量化真正的净获益,确保技术成就转化为有意义的临床效用。

第三,我们如何无缝安全地将经过验证的AI能力整合到真实的临床工作流程中,以实现真正的人机协作?这个问题推动了Sono-Agent的开发,这是一个人机协作工作流程原型,包含实时动态扫描辅助和知识增强的智能报告生成。该原型体现了上下文感知、非干扰性和可信赖性的设计原则,同时通过创新的事实核查机制解决了医学内容生成中AI幻觉等关键挑战。

使用精心构建的包含4,847个检查病例的多中心数据集进行的实验验证表明,USANet在所有评估指标上都取得了优于专用单任务模型的性能。在胃癌评估方面,统一框架在病灶定位方面达到了0.847的平均精度均值(mAP),在分割方面达到了0.831的Dice系数,在良恶性分类方面达到了0.912的ROC曲线下面积(AUC),与基线方法相比提高了15-20\%。在胰腺癌评估方面,尽管存在固有的可视化挑战,USANet仍达到了0.789的mAP、0.793的Dice系数和0.887的AUC,在病灶部分被遮挡的病例中表现尤为突出,这些情况下上下文信息证明至关重要。

临床验证表明AI衍生评估与病理分期信息之间存在强相关性,胃癌T分期的相关系数为0.78,胰腺癌分期的相关系数为0.72(p<0.001)。决策曲线分析显示在临床相关的决策阈值范围内具有显著的净获益,AI辅助在高风险场景(如手术候选资格评估)中提供最大获益。验证确认了统一框架提供的真正临床价值证明了实施努力和资源投资的合理性。

涉及不同经验水平医生的综合可用性评估表明,当AI辅助可用时,诊断准确性、检查效率和医生信心都有显著改善。诊断准确性的改善从有经验医生的7.5个百分点到新手医生的13.4个百分点不等,检查时间减少12-18\%,同时保持或改善诊断质量。重要的是,评估揭示了教育效应,医生即使在随后不使用AI支持的情况下也表现出改善的诊断技能,这表明除了即时帮助之外还有更广泛的益处。

本研究的理论贡献超越了超声成像,涉及医学AI架构设计的基本问题。"四大协同支柱"框架为利用解剖相关器官之间的自然关系提供了原则,而两阶段训练策略为具有稀缺标注数据但丰富无标注数据的领域提供了实用方法。全面的验证方法论通过优先考虑临床效用而非纯技术指标,解决了医学AI评估中的关键空白。

实际贡献包括首个使用经腹超声进行胃癌和胰腺癌联合评估的统一计算框架、在动态超声检查期间实时AI辅助的验证算法,以及通过知识增强机制在医学文档中安全整合生成式AI的实证方法。Sono-Agent原型为增强而非替代医生能力的人机协作工作流程提供了模板,同时保持临床自主性和信任。

本研究证明人工智能能够有效解决经腹超声在癌症评估中的根本局限性,同时尊重临床实践中固有的复杂性和细致入微之处。统一框架、全面验证方法论和实用工作流程整合方法的成功开发为AI辅助医学成像的更广泛应用提供了基础。强调协作而非替代人类判断的"协同智能"愿景已被证明在技术上可行且在临床上有价值,指向了能够实现AI在医疗保健中变革潜力的未来发展,同时保持定义优秀临床护理的基本人文要素。

这项工作的最终意义在于其通过胃癌和胰腺癌的早期检测以及超声技术在临床实践中的更有效利用来改善患者预后的潜在贡献。该研究提供了令人信服的证据,表明人工智能与医学实践的整合能够增强而非削弱人类能力,为通过深度理解临床需求和约束指导的创新技术方法解决当代肿瘤学中最具挑战性的问题之一提供了希望。