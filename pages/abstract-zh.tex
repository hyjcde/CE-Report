% !TEX root = ../thesis.tex

胃癌与胰腺癌因其高致死率和低早期诊断率,构成了全球性的重大健康挑战。经腹超声检查(Transabdominal Ultrasound, TAUS)凭借其安全无创、便捷经济、实时动态及宽泛解剖覆盖的显著优势,被认为是填补一线筛查技术鸿沟的最具潜力的候选者。然而,TAUS存在一个根本性的悖论:其最高的筛查潜力,却受限于最低的诊断一致性。这种诊断效能对操作者个人经验的深度依赖,以及对早期病灶的低敏感度和伪影干扰,严重阻碍了其潜力的完全释放。本研究的核心问题是:我们如何能够系统性地缓解这些固有的局限性,从而解锁TAUS的全部潜力?

本研究提出,解决之道在于构建一种"协同智能"(Synergistic Intelligence)的新范式,特别是通过一个专为胃与胰腺联合评估设计的统一计算框架。这一思路的理论基石是"四大协同支柱":解剖邻近性、病理相互依赖性、整合的临床工作流以及上下文AI分析的潜力。本论文的核心贡献在于设计、实现并验证一个统一的深度学习计算框架(A Unified Framework)。该框架旨在超越传统单任务模型的局限,能够在一个模型内同时学习并执行对胃癌与胰腺癌的多种评估任务,包括病灶检测、良恶性鉴别、以及与临床分期相关的关键影像特征评估。为实现这一目标,本研究将提出一种创新的"两阶段知识注入"训练策略,以最大化异构数据资产的协同效用。

为确保研究的临床转化价值,本论文将建立一个超越传统技术指标的多维度综合验证体系。该体系不仅包含技术层面的性能评估,更将AI模型的输出与最权威的临床终点——病理学金标准与术后TNM分期——进行直接的定量关联性分析,并采用决策曲线分析(DCA)来量化其在辅助临床决策中的净获益。最后,为解决"最后一公里"的应用落地问题,本研究将探索并构建一个名为"Sono-Agent"的人机协同工作流原型。该原型旨在将经过验证的统一框架能力,以一种实时、智能、非干扰的方式整合进动态的超声扫查与报告流程中,实现从被动图像分析到主动认知辅助的范式转变。

本研究的预期成果包括:一个性能优越的、基于协同效应理论的统一AI模型、一套严谨的临床价值验证方法论、以及一个人机协同工作流的设计范式。这些成果有望为提升经腹超声在胃癌与胰腺癌早期诊断中的作用提供一套完整的系统性解决方案,最终通过增强而非取代人类专家的智慧,赋能临床医生,改善患者预后。

\textbf{关键词:} 协同智能;经腹超声;深度学习;统一框架;胃癌;胰腺癌;临床验证;人机协同