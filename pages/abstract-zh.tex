% !TEX root = ../thesis.tex

本研究旨在解决大型语言模型(LLM)在应用于物理世界决策时面临的根本性"认知-物理鸿沟"。为应对此挑战,我们提出并计划实现一个名为CORTEX的新型Agent架构。

\textbf{研究背景与问题:} 尽管大型语言模型在文本推理方面取得了巨大成功,但当应用于需要与动态物理环境交互的任务时,存在着根本性的认知-物理鸿沟。这种鸿沟源于LLM基于静态文本语料库训练,缺乏对实时物理状态的情境化理解,导致其内部世界模型与物理现实脱节。

\textbf{研究目标与方法:} 本研究的核心贡献包含三个层面:1) 理论层面,我们拟构建一个"三层数字孪生决策框架"(L1-描述性,L2-预测性,L3-交互式),为系统性地评估物理世界AI提供理论基座;2) 架构层面,我们计划设计CORTEX架构,通过深度扩展RAG和Agent范式,系统性地解决LLM在物理世界中的现实接地、模型利用和安全执行三大挑战;3) 实证层面,我们提出"认知增益"的量化评估方法,并计划通过三个分别对应L1、L2、L3的代表性案例进行验证。

\textbf{研究进展与计划:} CORTEX架构基于认知科学启发的四阶段循环:1) 感知接地与情境构建,2) 因果推理与预测仿真,3) 行动策略生成与验证,4) 物理交互与模型校准。为验证该架构在不同领域的有效性,本研究采用多案例研究方法。

\textbf{已完成工作:} 建筑健康监测的预测决策案例研究已成功完成。该工作开发并验证了融合建筑信息模型(BIM)数据与实时传感器时序数据的数字孪生,证明了CORTEX架构能够显著提升维护决策质量,相比传统方法减少35\%的误报率。

\textbf{进行中的工作:} 目前正在开展两个额外的案例研究。第二个案例研究聚焦于医疗超声诊断中的辅助决策(癌症治疗规划),计划实现基于2D超声图像特征提取的非视觉数字孪生。第三个案例研究涉及无人机探索中的自主决策,拟利用实时3D点云数据构建数字孪生进行未知环境的导航和避障。

\textbf{预期成果与贡献:} 研究完成后,预期将定量验证CORTEX架构在多样化物理交互任务中显著提升LLM驱动决策的质量、鲁棒性和安全性。主要贡献包括:(1) 连接LLM推理与物理世界交互的理论框架,(2) 数字孪生增强认知架构的实用实现方法,(3) 跨三个不同领域的实证验证,(4) 此类系统在实际应用中的部署指南。

本研究成果预期将为开发更强大、更可靠的物理世界人工智能系统提供一个经过验证的、可扩展的架构蓝图,推动下一代AI系统向能够与物理世界进行可靠、智能交互的方向发展。