% !TEX root = ../thesis.tex

本研究旨在解决大型语言模型(LLM)在应用于物理世界决策时面临的根本性"认知-物理鸿沟"。尽管LLM在文本推理方面取得巨大成功,但在需要与动态物理环境交互的任务中,由于基于静态文本语料库训练而缺乏对实时物理状态的情境化理解,导致其内部世界模型与物理现实脱节。

为应对此挑战,本研究提出并计划实现名为CORTEX的新型Agent架构。研究的核心贡献体现在三个层面:理论层面构建"三层数字孪生决策框架"(L1-描述性,L2-预测性,L3-交互式),为系统性评估物理世界AI提供理论基座;架构层面设计CORTEX架构,通过深度扩展RAG和Agent范式,系统性解决LLM在物理世界中的现实接地、模型利用和安全执行三大挑战;实证层面提出"认知增益"量化评估方法,通过三个分别对应L1、L2、L3的代表性案例进行验证。

CORTEX架构基于认知科学启发的四阶段循环:感知接地与情境构建、因果推理与预测仿真、行动策略生成与验证、物理交互与模型校准。为验证该架构在不同领域的有效性,本研究采用多案例研究方法。建筑健康监测的预测决策案例研究已成功完成,开发并验证了融合建筑信息模型数据与实时传感器时序数据的数字孪生,证明CORTEX架构能够显著提升维护决策质量。

目前正在开展两个额外案例研究:医疗超声诊断中的辅助决策,计划实现基于2D超声图像特征提取的非视觉数字孪生;无人机探索中的自主决策,拟利用实时3D点云数据构建数字孪生进行未知环境导航和避障。研究完成后,预期将定量验证CORTEX架构在多样化物理交互任务中显著提升LLM驱动决策的质量、鲁棒性和安全性。本研究成果将为开发更强大、更可靠的物理世界人工智能系统提供经过验证的、可扩展的架构蓝图。