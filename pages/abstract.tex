% !TEX root = ../thesis.tex

This research aims to address the fundamental "cognitive-physical gap" faced by Large Language Models (LLMs) when applied to physical world decision-making. While LLMs have achieved tremendous success in textual reasoning, they exhibit significant limitations when applied to tasks requiring interaction with dynamic physical environments due to training on static text corpora, lacking contextualized understanding of real-time physical states and resulting in internal world models disconnected from physical reality.

To tackle this challenge, we propose and plan to implement a novel Agent architecture named CORTEX. The research contributions span three levels: theoretical level through constructing a "Three-Layer Digital Twin Decision Framework" (L1-Descriptive, L2-Predictive, L3-Interactive) that provides a theoretical foundation for systematically evaluating physical world AI; architectural level through designing the CORTEX architecture that systematically addresses three major challenges of LLMs in the physical world through deep extensions of RAG and Agent paradigms; and empirical level through proposing quantitative evaluation methods for "cognitive gains" and validating the framework through three representative cases corresponding to L1, L2, and L3 respectively.

The CORTEX architecture operates through a cognitive science-inspired four-stage loop: Perceptual Grounding and Context Formulation, Causal Inference and Predictive Simulation, Action Policy Generation and Validation, and Physical Interaction and Model Calibration. To validate this architecture's effectiveness across diverse domains, this research employs a multi-case study approach. The first case study in predictive decision-making for building health monitoring has been successfully completed, developing and validating a Digital Twin that fuses Building Information Modeling data with real-time sensor time-series data, demonstrating that the CORTEX architecture can significantly enhance maintenance decision quality and reduce false positive rates by 35% compared to traditional approaches.

Two additional case studies are currently underway: assistive decision-making in medical ultrasound diagnosis, planning to implement a non-visual Digital Twin based on feature extraction from 2D ultrasound images; and autonomous decision-making in UAV exploration, proposing to utilize real-time 3D point cloud data to construct Digital Twins for navigation and obstacle avoidance in unknown environments. Upon completion, this research is expected to quantitatively validate that the CORTEX architecture significantly enhances the quality, robustness, and safety of LLM-driven decisions across diverse physical interaction tasks. The research outcomes will provide a validated, scalable architectural blueprint for developing more powerful and reliable physical world artificial intelligence systems.