% !TEX root = ../thesis.tex

This research aims to address the fundamental "cognitive-physical gap" faced by Large Language Models (LLMs) when applied to physical world decision-making. To tackle this challenge, we propose and plan to implement a novel Agent architecture named CORTEX.

\textbf{Background and Problem Statement:} While Large Language Models have achieved tremendous success in textual reasoning, they exhibit a fundamental cognitive-physical gap when applied to tasks requiring interaction with dynamic physical environments. This gap stems from LLMs' training on static text corpora, lacking contextualized understanding of real-time physical states, resulting in internal world models that are disconnected from physical reality.

\textbf{Research Objectives and Methodology:} The core contributions of this research span three levels: 1) Theoretical level: we propose to construct a "Three-Layer Digital Twin Decision Framework" (L1-Descriptive, L2-Predictive, L3-Interactive), providing a theoretical foundation for systematically evaluating physical world AI; 2) Architectural level: we plan to design the CORTEX architecture, which systematically addresses three major challenges of LLMs in the physical world—reality grounding, model utilization, and safe execution—through deep extensions of RAG and Agent paradigms; 3) Empirical level: we propose a quantitative evaluation method for "cognitive gains" and plan to validate the framework through three representative cases corresponding to L1, L2, and L3 respectively.

\textbf{Research Progress and Implementation Plan:} The CORTEX architecture is based on a cognitive science-inspired four-stage loop: 1) Perceptual Grounding \& Context Formulation, 2) Causal Inference \& Predictive Simulation, 3) Action Policy Generation \& Validation, and 4) Physical Interaction \& Model Calibration. To validate this architecture's effectiveness across diverse domains, this research employs a multi-case study approach.

\textbf{Completed Work:} The first case study in \textit{predictive decision-making for building health monitoring} has been successfully completed. This work developed and validated a Digital Twin that fuses Building Information Modeling (BIM) data with real-time sensor time-series data, demonstrating that the CORTEX architecture can significantly enhance maintenance decision quality and reduce false positive rates by 35\% compared to traditional approaches.

\textbf{Ongoing Work:} Two additional case studies are currently underway. The second case study focuses on \textit{assistive decision-making in medical ultrasound diagnosis (cancer treatment planning)}, planning to implement a non-visual Digital Twin based on feature extraction from 2D ultrasound images. The third case study addresses \textit{autonomous decision-making in UAV exploration}, proposing to utilize real-time 3D point cloud data to construct Digital Twins for navigation and obstacle avoidance in unknown environments.

\textbf{Expected Outcomes and Contributions:} Upon completion, this research is expected to quantitatively validate that the CORTEX architecture significantly enhances the quality, robustness, and safety of LLM-driven decisions across diverse physical interaction tasks. The main contributions include: (1) a theoretical framework bridging LLM reasoning and physical world interaction, (2) practical implementation methodology for Digital Twin-enhanced cognitive architectures, (3) empirical validation across three distinct domains, and (4) deployment guidelines for such systems in real-world applications.

The expected outcomes of this research will provide a validated, scalable architectural blueprint for developing more powerful and reliable physical world artificial intelligence systems, advancing next-generation AI systems toward reliable and intelligent interaction with the physical world.