% !TEX root = ../thesis.tex

The rapid advancement of Large Language Models (LLMs) has opened unprecedented opportunities for automated decision-making and complex task planning. However, a critical gap remains when these models are applied to tasks requiring interaction with dynamic physical environments: LLMs exhibit a fundamental disconnect between their abstract reasoning processes and the concrete realities of the physical world. This limitation stems from LLMs' reliance on static, offline text corpora, resulting in internal world models that are inherently disembodied and lack contextual grounding in real-time physical states.

To systematically address this challenge, this thesis proposes the development of CORTEX (Cognitive Reasoning and Task EXecution architecture), a novel cognitive architecture that fundamentally reframes LLMs from standalone reasoning engines into "cognitive cores" that depend on dynamic world representations for effective decision-making. The key innovation lies in integrating Digital Twins (DTs) as vehicles for world representation, functionally defined as computational models that dynamically represent physical systems with sufficient fidelity to support task-specific decision-making requirements.

\textbf{Research Progress and Methodology:} The CORTEX architecture operates through a cognitive science-inspired, four-stage loop: 1) Perceptual Grounding \& Context Formulation, 2) Causal Inference \& Predictive Simulation, 3) Action Policy Generation \& Validation, and 4) Physical Interaction \& Model Calibration. To validate this architecture's effectiveness across diverse domains, this research employs a multi-case study approach spanning three distinct application areas.

\textbf{Completed Work:} The first case study in \textit{predictive decision-making for building health monitoring} has been successfully completed. This work developed and validated a DT that fuses Building Information Modeling (BIM) data with real-time sensor time-series data, demonstrating the CORTEX architecture's capability to enhance maintenance decision quality and reduce false positive rates by 35\% compared to traditional approaches.

\textbf{Current Work:} Two additional case studies are currently underway. The second case study focuses on \textit{assistive decision-making in medical ultrasound diagnosis}, implementing a non-visual DT composed of extracted features from 2D ultrasound images. Preliminary results show promising improvements in diagnostic accuracy and confidence scoring. The third case study addresses \textit{autonomous decision-making in UAV exploration}, utilizing a DT built from real-time 3D point cloud data for navigation and obstacle avoidance in unknown environments.

\textbf{Expected Outcomes:} Upon completion, this research will demonstrate that the CORTEX architecture significantly enhances the quality, robustness, and safety of LLM-driven decisions across diverse physical interaction tasks. The expected contributions include: (1) a theoretical framework bridging LLM reasoning and physical world interaction, (2) a practical implementation methodology for Digital Twin-enhanced cognitive architectures, (3) empirical validation across three distinct domains, and (4) guidelines for deploying such systems in real-world applications.

\textbf{Timeline and Feasibility:} With one case study completed and two underway, the research is progressing according to schedule. The diverse domain validation demonstrates the architecture's generalizability while the successful completion of the building monitoring case study provides proof of concept for the remaining work. This research is positioned to make significant contributions toward next-generation AI systems capable of reliable and intelligent interaction with the physical world.