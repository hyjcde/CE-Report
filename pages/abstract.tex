% !TEX root = ../thesis.tex

The rise of Large Language Models (LLMs) has introduced revolutionary potential for automated decision-making and complex task planning. However, when applied to tasks requiring interaction with the dynamic and unstructured physical world, LLMs exhibit a fundamental shortcoming: a profound decoupling between their abstract reasoning processes and the physical reality they must operate within. As LLMs derive their knowledge from static, offline text corpora, their internal world models are inherently disembodied and a-contextual, leading to plans and decisions that are often disconnected from specific physical constraints, spatio-temporal dynamics, and the real-time state of a task.

To systematically address this issue, this thesis proposes and develops a novel cognitive architecture named CORTEX (Cognitive Reasoning and Task EXecution architecture). The core idea of CORTEX is a paradigm shift: it reframes the LLM from a simple reasoning engine into a "cognitive core" that is critically dependent on a dynamic world representation to think effectively. To this end, CORTEX introduces the Digital Twin (DT) as the vehicle for this representation, functionally defining it as any computational model that dynamically represents a physical system with a level of fidelity sufficient to support the decision-making needs of the task at hand. This definition allows the DT to encompass a wide range of forms, from high-fidelity 3D models to multi-dimensional feature-space models.

The CORTEX architecture operates through a cognitive science-inspired, four-stage loop: 1) Perceptual Grounding \& Context Formulation, where the LLM queries the DT to understand the present state; 2) Causal Inference \& Predictive Simulation, where the LLM manipulates the DT to explore future consequences; 3) Action Policy Generation \& Validation, where the LLM verifies and refines action plans within the DT; and 4) Physical Interaction \& Model Calibration, where feedback from the physical world updates the DT, closing the learning loop.

To validate the architecture's effectiveness and versatility, this work presents three cross-domain case studies: 1) predictive decision-making in building health monitoring, using a DT that fuses BIM and time-series data; 2) assistive decision-making in medical ultrasound diagnosis, using a non-visual DT composed of features from 2D images; and 3) autonomous decision-making in UAV exploration, using a DT built from real-time 3D point clouds. The results demonstrate that the CORTEX architecture significantly enhances the quality, robustness, and safety of LLM-driven decisions in physical interaction tasks, paving a solid theoretical and technical path toward next-generation AI systems capable of reliable and intelligent interaction with the physical world.