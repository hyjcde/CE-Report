% !TEX root = ../thesis.tex

Gastric and pancreatic cancers, with their exceptionally high mortality rates and persistently low early detection rates, constitute a major global health challenge that demands innovative approaches to diagnostic technology. Despite decades of medical advancement, five-year survival rates for these devastating diseases remain stubbornly low, with gastric cancer survival below 35\% and pancreatic cancer survival under 12\%, largely due to the insidious nature of early-stage disease that often escapes detection until advanced stages when therapeutic options become limited. The fundamental challenge lies not in treating advanced disease, which has reached therapeutic ceilings despite sophisticated interventions, but in achieving genuine early detection when curative interventions remain viable and patient prognosis is most favorable.

Transabdominal ultrasound (TAUS), with its remarkable advantages of safety, non-invasiveness, convenience, cost-effectiveness, real-time dynamic assessment, and broad anatomical coverage, emerges as the most promising candidate for filling the critical screening gap in first-line diagnostic technologies. However, TAUS presents a fundamental paradox that has limited its clinical utility: an imaging modality with the highest screening potential is constrained by the lowest diagnostic consistency. This diagnostic paradox, characterized by extreme operator dependence and low sensitivity for early lesions complicated by artifact interference, severely limits the full realization of its tremendous potential. The core question driving this research is: How can we systematically mitigate these inherent limitations to unlock the full potential of TAUS through innovative technological approaches?

This dissertation addresses this challenge through the theoretical framework of "synergistic intelligence," recognizing that gastric and pancreatic cancers possess inseparable intrinsic connections in anatomy, pathology, and clinical workflow that can be leveraged through unified computational approaches. Traditional "single-task, single-model" research paradigms artificially fragment these natural relationships, missing opportunities for contextual analysis and cross-organ information utilization. This research proposes that artificial intelligence, specifically through unified multi-task learning architectures, provides an unprecedented opportunity to capture and utilize these synergistic effects while reducing dependence on individual operator experience.

The research systematically addresses three interconnected scientific questions through comprehensive experimental investigation and clinical validation. First, how can we construct a unified computational framework that effectively leverages synergistic effects between stomach and pancreas? This question led to the development of USANet (Unified Sonographic Assessment Network), a novel deep learning architecture employing CNN-Transformer hybrid design coupled with an innovative "Two-Stage Knowledge Injection" training strategy. The first stage utilizes large-scale unlabeled abdominal ultrasound data for general anatomical representation learning through self-supervised learning approaches, while the second stage performs multi-task knowledge joint fine-tuning using precisely annotated gastric and pancreatic cancer datasets.

Second, how can we scientifically and rigorously validate the clinical value of this framework to establish clinical trust? This question motivated the establishment of a comprehensive "technical-clinical-decision" tripartite validation system that transcends traditional technical metrics to encompass clinical endpoint correlation and decision-making value assessment. The validation framework directly links AI model outputs with pathological gold standards and employs decision curve analysis to quantify genuine net benefit from a clinical decision-making perspective, ensuring that technical achievements translate into meaningful clinical utility.

Third, how can we seamlessly and safely integrate validated AI capabilities into real clinical workflows to achieve genuine human-AI collaboration? This question drove the development of Sono-Agent, a human-AI collaborative workflow prototype incorporating real-time dynamic scanning assistance and knowledge-enhanced intelligent report generation. The prototype embodies design principles of context-awareness, non-intrusiveness, and trustworthiness while addressing critical challenges such as AI hallucination in medical content generation through innovative fact-checking mechanisms.

Experimental validation using a meticulously constructed multi-center dataset of 4,847 examination cases demonstrated that USANet achieved superior performance compared to specialized single-task models across all evaluation metrics. For gastric cancer assessment, the unified framework achieved mean Average Precision (mAP) of 0.847 for lesion localization, Dice coefficients of 0.831 for segmentation, and Area Under the ROC Curve (AUC) of 0.912 for benign-malignant classification, representing improvements of 15-20% over baseline approaches. For pancreatic cancer assessment, despite inherent visualization challenges, USANet achieved mAP of 0.789, Dice coefficients of 0.793, and AUC of 0.887, with particularly notable improvements in cases with partial lesion obscuration where contextual information proved crucial.

The clinical validation demonstrated strong correlations between AI-derived assessments and pathological staging information, with correlation coefficients of 0.78 for gastric cancer T-staging and 0.72 for pancreatic cancer staging (p<0.001). Decision curve analysis revealed significant net benefits across clinically relevant decision thresholds, with AI assistance providing maximum benefit in high-stakes scenarios such as surgical candidacy assessment. The validation confirmed that the unified framework provides genuine clinical value that justifies implementation efforts and resource investments.

Comprehensive usability evaluation involving physicians across different experience levels demonstrated significant improvements in diagnostic accuracy, examination efficiency, and physician confidence when AI assistance was available. Diagnostic accuracy improvements ranged from 7.5 percentage points for experienced physicians to 13.4 percentage points for novice physicians, with examination time reductions of 12-18% while maintaining or improving diagnostic quality. Importantly, the evaluation revealed educational effects where physicians demonstrated improved diagnostic skills even when subsequently working without AI support, suggesting broader benefits beyond immediate assistance.

The theoretical contributions of this research extend beyond ultrasound imaging to fundamental questions of medical AI architecture design. The "Four Pillars of Synergy" framework provides principles for leveraging natural relationships between anatomically related organs, while the two-stage training strategy offers a practical approach for domains with scarce annotated data but abundant unlabeled data. The comprehensive validation methodology addresses critical gaps in medical AI evaluation by prioritizing clinical utility over purely technical metrics.

The practical contributions include the first unified computational framework for joint gastric and pancreatic cancer assessment using transabdominal ultrasound, validated algorithms for real-time AI assistance during dynamic ultrasound examination, and demonstrated approaches for safe integration of generative AI in medical documentation through knowledge enhancement mechanisms. The Sono-Agent prototype provides a template for human-AI collaborative workflows that enhance rather than replace physician capabilities while maintaining clinical autonomy and trust.

This research demonstrates that artificial intelligence can effectively address fundamental limitations of transabdominal ultrasound in cancer assessment while respecting the complexity and nuance inherent in clinical practice. The successful development of unified frameworks, comprehensive validation methodologies, and practical workflow integration approaches provides a foundation for broader applications of AI-assisted medical imaging. The vision of "synergistic intelligence" emphasizing collaboration rather than replacement of human judgment has proven both technically feasible and clinically valuable, pointing toward future developments that can realize the transformative potential of AI in healthcare while preserving essential human elements that define excellent clinical care.

The ultimate significance of this work lies in its potential contribution to improving patient outcomes through earlier detection of gastric and pancreatic cancers and more effective utilization of ultrasound technology in clinical practice. The research provides compelling evidence that the integration of artificial intelligence into medical practice can enhance rather than diminish human capabilities, offering hope for addressing one of the most challenging problems in contemporary oncology through innovative technological approaches guided by deep understanding of clinical needs and constraints.