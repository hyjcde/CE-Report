% !TEX root = ../thesis.tex

Gastric and pancreatic cancers, with their exceptionally high mortality rates and persistently low early detection rates, constitute a major global health challenge. Transabdominal ultrasound (TAUS), with its remarkable advantages of safety, non-invasiveness, convenience, cost-effectiveness, real-time dynamic assessment, and broad anatomical coverage, is considered the most promising candidate for filling the screening gap in first-line diagnostic technologies. However, TAUS presents a fundamental paradox: an imaging modality with the highest screening potential is constrained by the lowest diagnostic consistency. This diagnostic paradox, characterized by extreme operator dependence and low sensitivity for early lesions with artifact interference, severely limits the full realization of its potential. The core question of this research is: How can we systematically mitigate these inherent limitations to unlock the full potential of TAUS?

This study proposes that the solution lies in constructing a new paradigm of "Synergistic Intelligence," specifically through a unified computational framework designed for joint assessment of gastric and pancreatic malignancies. The theoretical foundation of this approach rests on the "Four Pillars of Synergy": anatomical adjacency, pathological interdependence, integrated clinical workflow, and the potential for contextual AI analysis. The core contribution of this thesis is the design, implementation, and validation of a Unified Framework for deep learning computation. This framework aims to transcend the limitations of traditional single-task models by learning and executing multiple assessment tasks for both gastric and pancreatic cancers within a single model, including lesion detection, benign-malignant differentiation, and evaluation of key imaging features related to clinical staging. To achieve this goal, this research proposes an innovative "Two-Stage Knowledge Injection" training strategy to maximize the synergistic utility of heterogeneous data assets.

To ensure the clinical translational value of the research, this thesis establishes a multi-dimensional comprehensive validation system that goes beyond traditional technical metrics. This system includes not only technical performance evaluation but also direct quantitative correlation analysis between AI model outputs and the most authoritative clinical endpoints—pathological gold standards and postoperative TNM staging—and employs Decision Curve Analysis (DCA) to quantify the net benefit in assisting clinical decision-making. Finally, to address the "last mile" application challenge, this research explores and constructs a human-AI collaborative workflow prototype called "Sono-Agent." This prototype aims to integrate the validated unified framework capabilities into dynamic ultrasound scanning and reporting processes in a real-time, intelligent, and non-intrusive manner, achieving a paradigmatic shift from passive image analysis to active cognitive assistance.

The expected outcomes of this research include: a superior unified AI model based on synergistic effects theory, a rigorous clinical value validation methodology, and a design paradigm for human-AI collaborative workflows. These outcomes are expected to provide a complete systematic solution for enhancing the role of transabdominal ultrasound in early diagnosis of gastric and pancreatic cancers, ultimately empowering clinicians and improving patient outcomes by augmenting rather than replacing human expert intelligence.

\textbf{Keywords:} Synergistic Intelligence; Transabdominal Ultrasound; Deep Learning; Unified Framework; Gastric Cancer; Pancreatic Cancer; Clinical Validation; Human-AI Collaboration